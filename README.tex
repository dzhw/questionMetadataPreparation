% Options for packages loaded elsewhere
\PassOptionsToPackage{unicode}{hyperref}
\PassOptionsToPackage{hyphens}{url}
%
\documentclass[
]{article}
\usepackage{lmodern}
\usepackage{amssymb,amsmath}
\usepackage{ifxetex,ifluatex}
\ifnum 0\ifxetex 1\fi\ifluatex 1\fi=0 % if pdftex
  \usepackage[T1]{fontenc}
  \usepackage[utf8]{inputenc}
  \usepackage{textcomp} % provide euro and other symbols
\else % if luatex or xetex
  \usepackage{unicode-math}
  \defaultfontfeatures{Scale=MatchLowercase}
  \defaultfontfeatures[\rmfamily]{Ligatures=TeX,Scale=1}
\fi
% Use upquote if available, for straight quotes in verbatim environments
\IfFileExists{upquote.sty}{\usepackage{upquote}}{}
\IfFileExists{microtype.sty}{% use microtype if available
  \usepackage[]{microtype}
  \UseMicrotypeSet[protrusion]{basicmath} % disable protrusion for tt fonts
}{}
\makeatletter
\@ifundefined{KOMAClassName}{% if non-KOMA class
  \IfFileExists{parskip.sty}{%
    \usepackage{parskip}
  }{% else
    \setlength{\parindent}{0pt}
    \setlength{\parskip}{6pt plus 2pt minus 1pt}}
}{% if KOMA class
  \KOMAoptions{parskip=half}}
\makeatother
\usepackage{xcolor}
\IfFileExists{xurl.sty}{\usepackage{xurl}}{} % add URL line breaks if available
\IfFileExists{bookmark.sty}{\usepackage{bookmark}}{\usepackage{hyperref}}
\hypersetup{
  hidelinks,
  pdfcreator={LaTeX via pandoc}}
\urlstyle{same} % disable monospaced font for URLs
\usepackage[margin=1in]{geometry}
\usepackage{color}
\usepackage{fancyvrb}
\newcommand{\VerbBar}{|}
\newcommand{\VERB}{\Verb[commandchars=\\\{\}]}
\DefineVerbatimEnvironment{Highlighting}{Verbatim}{commandchars=\\\{\}}
% Add ',fontsize=\small' for more characters per line
\usepackage{framed}
\definecolor{shadecolor}{RGB}{248,248,248}
\newenvironment{Shaded}{\begin{snugshade}}{\end{snugshade}}
\newcommand{\AlertTok}[1]{\textcolor[rgb]{0.94,0.16,0.16}{#1}}
\newcommand{\AnnotationTok}[1]{\textcolor[rgb]{0.56,0.35,0.01}{\textbf{\textit{#1}}}}
\newcommand{\AttributeTok}[1]{\textcolor[rgb]{0.77,0.63,0.00}{#1}}
\newcommand{\BaseNTok}[1]{\textcolor[rgb]{0.00,0.00,0.81}{#1}}
\newcommand{\BuiltInTok}[1]{#1}
\newcommand{\CharTok}[1]{\textcolor[rgb]{0.31,0.60,0.02}{#1}}
\newcommand{\CommentTok}[1]{\textcolor[rgb]{0.56,0.35,0.01}{\textit{#1}}}
\newcommand{\CommentVarTok}[1]{\textcolor[rgb]{0.56,0.35,0.01}{\textbf{\textit{#1}}}}
\newcommand{\ConstantTok}[1]{\textcolor[rgb]{0.00,0.00,0.00}{#1}}
\newcommand{\ControlFlowTok}[1]{\textcolor[rgb]{0.13,0.29,0.53}{\textbf{#1}}}
\newcommand{\DataTypeTok}[1]{\textcolor[rgb]{0.13,0.29,0.53}{#1}}
\newcommand{\DecValTok}[1]{\textcolor[rgb]{0.00,0.00,0.81}{#1}}
\newcommand{\DocumentationTok}[1]{\textcolor[rgb]{0.56,0.35,0.01}{\textbf{\textit{#1}}}}
\newcommand{\ErrorTok}[1]{\textcolor[rgb]{0.64,0.00,0.00}{\textbf{#1}}}
\newcommand{\ExtensionTok}[1]{#1}
\newcommand{\FloatTok}[1]{\textcolor[rgb]{0.00,0.00,0.81}{#1}}
\newcommand{\FunctionTok}[1]{\textcolor[rgb]{0.00,0.00,0.00}{#1}}
\newcommand{\ImportTok}[1]{#1}
\newcommand{\InformationTok}[1]{\textcolor[rgb]{0.56,0.35,0.01}{\textbf{\textit{#1}}}}
\newcommand{\KeywordTok}[1]{\textcolor[rgb]{0.13,0.29,0.53}{\textbf{#1}}}
\newcommand{\NormalTok}[1]{#1}
\newcommand{\OperatorTok}[1]{\textcolor[rgb]{0.81,0.36,0.00}{\textbf{#1}}}
\newcommand{\OtherTok}[1]{\textcolor[rgb]{0.56,0.35,0.01}{#1}}
\newcommand{\PreprocessorTok}[1]{\textcolor[rgb]{0.56,0.35,0.01}{\textit{#1}}}
\newcommand{\RegionMarkerTok}[1]{#1}
\newcommand{\SpecialCharTok}[1]{\textcolor[rgb]{0.00,0.00,0.00}{#1}}
\newcommand{\SpecialStringTok}[1]{\textcolor[rgb]{0.31,0.60,0.02}{#1}}
\newcommand{\StringTok}[1]{\textcolor[rgb]{0.31,0.60,0.02}{#1}}
\newcommand{\VariableTok}[1]{\textcolor[rgb]{0.00,0.00,0.00}{#1}}
\newcommand{\VerbatimStringTok}[1]{\textcolor[rgb]{0.31,0.60,0.02}{#1}}
\newcommand{\WarningTok}[1]{\textcolor[rgb]{0.56,0.35,0.01}{\textbf{\textit{#1}}}}
\usepackage{graphicx}
\makeatletter
\def\maxwidth{\ifdim\Gin@nat@width>\linewidth\linewidth\else\Gin@nat@width\fi}
\def\maxheight{\ifdim\Gin@nat@height>\textheight\textheight\else\Gin@nat@height\fi}
\makeatother
% Scale images if necessary, so that they will not overflow the page
% margins by default, and it is still possible to overwrite the defaults
% using explicit options in \includegraphics[width, height, ...]{}
\setkeys{Gin}{width=\maxwidth,height=\maxheight,keepaspectratio}
% Set default figure placement to htbp
\makeatletter
\def\fps@figure{htbp}
\makeatother
\setlength{\emergencystretch}{3em} % prevent overfull lines
\providecommand{\tightlist}{%
  \setlength{\itemsep}{0pt}\setlength{\parskip}{0pt}}
\setcounter{secnumdepth}{-\maxdimen} % remove section numbering

\author{}
\date{\vspace{-2.5em}}

\begin{document}

\href{https://travis-ci.org/dzhw/questionMetadataPreparation}{\includegraphics{https://travis-ci.org/dzhw/questionMetadataPreparation.svg?branch=master}}
\href{https://codecov.io/github/dzhw/questionMetadataPreparation?branch=master}{\includegraphics{https://codecov.io/github/dzhw/questionMetadataPreparation/branch/master/graph/badge.svg}}
\href{https://www.tidyverse.org/lifecycle/\#stable}{\includegraphics{https://img.shields.io/badge/lifecycle-stable-brightgreen.svg}}
\href{https://dzhw.github.io/questionMetadataPreparation/}{\includegraphics{https://img.shields.io/badge/documentation--brightgreen}}

\hypertarget{question-metadata-preparation}{%
\section{\texorpdfstring{\href{https://dzhw.github.io/questionMetadataPreparation/}{Question
Metadata
Preparation}}{Question Metadata Preparation}}\label{question-metadata-preparation}}

This \href{https://www.r-project.org/about.html}{R} package
(\href{https://dzhw.github.io/questionMetadataPreparation/}{Question
Metadata Preparation}) helps preparing question-metadata for the
\href{https://metadata.fdz.dzhw.eu}{MDM} of the research data center of
the dzhw. If you do not work for the research data center of the dzhw,
this package will probably be only useful for learning purposes, as it
is specifically designed to help with our internal processes.

\hypertarget{installation-for-users}{%
\subsection{Installation for Users}\label{installation-for-users}}

If you are a Windows user you might need to install the
\href{https://cran.r-project.org/bin/windows/Rtools/}{Rtools}, in case
one of the dependencies is not available as a binary. Make sure to use
the correct -- matching to the version of your R installation --
version, not necessarily the newest version.

You can install the released version of questionMetadataPreparation from
\href{https://github.com/dzhw/questionMetadataPreparation}{Github}
within your \href{https://www.r-project.org/about.html}{R} session:

\begin{Shaded}
\begin{Highlighting}[]
\KeywordTok{install.packages}\NormalTok{(}\StringTok{"remotes"}\NormalTok{, }\DataTypeTok{dependencies =} \OtherTok{TRUE}\NormalTok{)}
\NormalTok{remotes}\OperatorTok{::}\KeywordTok{install\_github}\NormalTok{(}\StringTok{"dzhw/questionMetadataPreparation"}\NormalTok{)}
\end{Highlighting}
\end{Shaded}

In order to convert a Zofar export into a format which can be manually
edited, you have to run:

\begin{Shaded}
\begin{Highlighting}[]
\KeywordTok{convert\_zofar\_export\_to\_handcrafted\_questionnaire}\NormalTok{(}\StringTok{"./questions/ins1"}\NormalTok{)}
\end{Highlighting}
\end{Shaded}

The output will be written to \texttt{"./handcrafted/questions"}.

A set of handcrafted questionnaires can be manually converted into the
MDM format by running

\begin{Shaded}
\begin{Highlighting}[]
\KeywordTok{convert\_handcrafted\_questionnaires\_to\_mdm\_format}\NormalTok{(}\StringTok{"./questions"}\NormalTok{)}
\end{Highlighting}
\end{Shaded}

The output will be written to \texttt{"./mdm/questions"}.

\hypertarget{development}{%
\subsection{Development}\label{development}}

Developers need to setup the R devtools on their machine.

\begin{Shaded}
\begin{Highlighting}[]
\KeywordTok{install.packages}\NormalTok{(}\StringTok{"devtools"}\NormalTok{, }\DataTypeTok{dependencies =} \OtherTok{TRUE}\NormalTok{)}
\NormalTok{devtools}\OperatorTok{::}\KeywordTok{install\_github}\NormalTok{(}\StringTok{"dzhw/questionMetadataPreparation"}\NormalTok{)}
\end{Highlighting}
\end{Shaded}

After setting up devtools you can install all required R packages with

\begin{Shaded}
\begin{Highlighting}[]
\ExtensionTok{R}\NormalTok{ {-}e }\StringTok{\textquotesingle{}devtools::install\_deps(dep = T)\textquotesingle{}}
\end{Highlighting}
\end{Shaded}

You can build the package on you local machine with

\begin{Shaded}
\begin{Highlighting}[]
\ExtensionTok{R}\NormalTok{ CMD build .}
\end{Highlighting}
\end{Shaded}

Before pushing to Github (and thus kicking of CI) you should run

\begin{Shaded}
\begin{Highlighting}[]
\ExtensionTok{R}\NormalTok{ CMD check *tar.gz}
\end{Highlighting}
\end{Shaded}

\hypertarget{usefull-links-further-documentation}{%
\section{Usefull links, further
documentation}\label{usefull-links-further-documentation}}

\begin{itemize}
\tightlist
\item
  \href{https://metadatamanagement.readthedocs.io/de/stable/questions.html}{readthedocs}
\item
  \href{https://dzhw.github.io/questionMetadataPreparation/index.html}{Github
  repository documentation} and the -
  \href{https://dzhw.github.io/questionMetadataPreparation/articles/question_metadata_preparation_introduction.html}{introduction
  how to use questionMetadata Preparation}
\item
  \href{https://dzhw.github.io/questionMetadataPreparation/articles/general_workflow_and_tips.html}{Further
  Documentation/FAQ}
\end{itemize}

\hypertarget{having-trouble}{%
\subsection{Having trouble?}\label{having-trouble}}

Please file an issue in our
\href{https://github.com/dzhw/metadatamanagement/issues}{issue tracker}

\end{document}
